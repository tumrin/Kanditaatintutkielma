\chapter{Johdanto} \label{Johdanto}
Tämän tutkielman tarkoitus on tutkia Rust-ohjelmointikielen soveltuvuutta C- ja \Cpp-ohjelmointikielien korvaajaksi erityisesti tilanteissa, joissa tarvitaan järjestelmätason kielille ominaista nopeaa suorituskykyä. Tutkielma käsittelee Rustin ominaisuuksia verrattuna C- ja \Cpp-ohjelmointikieliin sekä esittelee niiden tyypillisiä ongelmia ja haasteita, joita Rustin käytöllä olisi mahdollista ratkaista. Tutkielmassa viitataan C- ja \Cpp-ohjelmointikieliin yhteisesti käsitteenä C-kielet, kun kuvataan molemmille yhteisiä ominaisuuksia.

Tutkimuskysymykset ovat seuraavat:
 \begin{enumerate}
  \item Sisältääkö Rust tarvittavat ominaisuudet C-kielien korvaamiseen?
  \item Mitä hyötyjä Rustin käytöllä saavutettaisiin?
  \item Mitkä ovat C-kielien suurimmat haasteet?
\end{enumerate}

Tutkielman tavoitteena on selvittää, onko Rust ominaisuuksiltaan verrattavissa C-kieliin ja voidaanko sen ominaisuuksia hyödyntämällä saavuttaa merkittäviä hyötyjä kehitettävien sovellusten vakaudessa ja turvallisuudessa. Tutkielma pyrkii myös tunnistamaan Rustin käytössä esiintyvät haasteet ja ehdottamaan niihin ratkaisuja.

Tutkimusmenetelmänä käytettiin pääosin kirjallisuuskatsausta ACM- ja IEEE-tietokannoista löytyvistä vertaisarvioiduista julkaisuista sekä ohjelmointikielien omasta dokumentaatiosta. Lisäksi tutkimuksessa on hyödynnetty itse tehtyjä ohjelmakoodiesimerkkejä ja suorituskykymittauksia.