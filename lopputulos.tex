\chapter{Loppupäätelmät} \label{Loppupäätelmät}

Tutkielma osoittaa, että Rust on ominaisuuksiltaan käytettävissä samanlaisissa sovelluksissa C-kielien kanssa. Lisäksi Rustin käytöllä voidaan poistaa monia vakavien haavoittuvuuksien ja ohjelmien epävakauden aiheuttajia. Rustin ominaisuudet voivat myös helpottaa kehitystyötä tarjoamalla helpon riippuvuuksienhallinnan sekä hyvät virheenjäljitystyökalut. Rust ei kuitenkaan poistaa vastuuta ohjelmoijalta kokonaan eikä estä ohjelmoijaa tekemästä virheitä, jotka vaikuttavat ohjelman turvallisuuteen ja vakauteen. Tästä huolimatta Rustin käytöllä saavutetut hyödyt ovat riittäviä, jotta sen käyttö C-kielien sijaan on perusteltua, mikäli jokin syy ei estä sen käyttöä.

Rustin käyttöönoton suurimmat haasteet ovat ekosysteemin nuoruus sekä osaamisen vähäisyys verrattuna C-kieliin. Tätä puutetta paikkaa kieleen vahvasti sidoksissa oleva pakettihallintaratkaisu kolmannen osapuolten kirjastoja ja riippuvuuksia varten sekä helposti saatavilla oleva ja kattava dokumentaatio ja opetusmateriaali. Rustin käyttöönottoa edesauttaa myös se, että sillä on useiden suurien yritysten tuki. Rustin kehityksen vastuu siirtyikin vuonna 2021 Mozillalta Rust Foundation-järjestölle, jonka perustajina ovat AWS, Huawei, Google, Microsoft ja Mozilla~\cite{rustfoundation}.

Tutkielman aihetta käsiteltiin vahvasti Rustin hyötyjen näkökulmasta, jolloin sen haasteet jäivät vähemmälle huomiolle. Samoin C-kieliä käsiteltiin niiden aiheuttamien haasteiden kautta, jolloin niihin vuosien aikana tehdyt turvallisuutta ja vakautta parantavat lisäykset jäivät vähemmälle huomiolle. Rustin käyttö ei ole ongelmatonta ja sen käytön yleistyessä ilmenee varmasti sille ominaisia ongelmia. Rustin haasteista ja sen ominaisuuksiin liittyvistä ongelmista olisi mahdollista tehdä jatkotutkimuksia. Lisäksi C-kielille kehitetyistä muistiturvallisuutta parantavista sovelluksista ja ohjelmointitekniikoista olisi mahdollista tehdä syvällisempiä jatkotutkimuksia. Myös Rustin vaikutuksista kehitysajan pituuteen voisi tehdä tutkimuksia, sillä se on yleinen kritiikki Rustia kohtaan.

Tutkielma esittää, että Rust olisi mahdollinen korvaaja C-kielille, mutta tämä ei tarkoita, että kaikki olemassa oleva C-kielillä kirjoitettu koodi pitäisi uudelleenkirjoittaa Rustilla. Sen sijaan Rustia tulisi harkita C-kielille vaihtoehtona uusia projekteja aloitettaessa, sekä silloin, kun lisätään ominaisuuksia C:llä tai \Cpp:lla kirjoitettuun ohjelmaan. Rust tarjoaa työkalut, joilla on mahdollista rakentaa uutta toiminnallisuutta C-kielillä kirjoitetun koodin päälle, mikä nähdään esimerkiksi Linux-ytimessä, jossa Rust on tarkoitus ottaa käyttöön toisena sallittuna kehityskielenä.