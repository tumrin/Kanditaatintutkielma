\chapter{Rust, C ja \Cpp} \label{esittely}
Tässä luvussa esitellään tutkimuksen kohteena olevat ohjelmointikielet sekä niiden historia, ominaisuudet ja käyttökohteet. Luvussa vertaillaan lisäksi C-kielien ominaisuuksia Rustiin ja osoitetaan, että Rust on ominaisuuksiltaan verrattavissa niihin.
\section{Historia}
C on vuonna 1972 Dennis Ritchien kehittämä järjestelmätason ohjelmointikieli. C kehitettiin UNIX-käyttöjärjestelmää ja sen sovelluksia varten. C-ohjelmointikielen käyttö levisi kuitenkin laajasti sen yksinkertaisuuden, nopeuden ja laitteistoriippumattomuuden takia.~\cite[pp.~ix--xi]{Cbook}

\Cpp -ohjelmointikielen kehitti Bjarne Stroustrup vuonna 1980, jolloin se tunnettiin vielä nimellä "C with classes"~eli C luokkien kanssa. Vuonna 1983 nimi muutettiin nykyiseen \Cpp-muotoon. \Cpp kehitettiin C-kielen pohjalta lisäämällä siihen luokkajärjestelmä, jonka avulla ohjelmoija voi itse määrittää uusia tyyppejä~\cite[pp.~v--xii]{Cppbook}.

Rust on Graydon Hoaren suunnittelema ja Mozillan kehittämä järjestelmätason ohjelmointikieli, joka pyrkii tarjoamaan korkean tason syntaksin ja matalan tason suorituskyvyn sekä hyödyntämään kääntäjän turvallisuusominaisuuksia muistinhallintaan ja samanaikaisuuteen liittyvien ongelmien poistamiseen~\cite{mozillarust}. Rust-ohjelmointikielen ensimmäinen vakaa versio julkaistiin vuonna 2015~\cite{rust1blog}.

\section{Käyttö}

C- ja \Cpp-kieliä käytetään moniin sovelluksiin, mutta nopeutensa vuoksi ne sopivat erityisesti käyttöjärjestelmiin ja laiteajureihin sekä 3D-sovelluksiin, kuten pelimoottoreihin ja mallinnusohjelmiin. Tunnettuja esimerkkejä C-kielillä kirjoitetuista sovelluksista ovat Linux-ydin, Blender ja Unreal engine. C-kielet ovat nopeutensa ja laiteläheisyytensä vuoksi omiaan myös sulautettuihin järjestelmiin ja reaaliaikaisiin käyttöjärjestelmiin.

Rust on huomattavasti C- ja \Cpp-kieliä uudempi, joten sen käyttö ei ole vielä yhtä vakiintunutta. Rust soveltuu nopeutensa puolesta kuitenkin samanlaisiin sovelluksiin kuin C-kielet ja molemmilla onkin yhteisiä käyttökohteita, kuten sulautetut järjestelmät ja WebAssembly~\cite{webassembly}. Esimerkkejä Rustilla kirjoitetuista sovelluksista ovat esimerkiksi Servo-selainmoottori ja SWC-niminen Javascript-kääntäjä. Rust soveltuu myös käyttöjärjestelmien kirjoittamiseen ja sillä onkin luotu Unixia mukaileva Redox OS. Lisäksi Rustia ollaan ottamassa käyttöön toiseksi viralliseksi kieleksi Linux-ytimeen~\cite{rustkernel}.

\section{Ominaisuudet}
Rustin täytyy vastata ominaisuuksiltaan C-kieliä voidakseen toimia korvaavana kielenä samanlaisissa sovelluksissa. Tässä luvussa käydään läpi C-kielien siirrettävyyttä, nopeutta, kolmannen osapuolten kirjastojen käyttöä sekä virheenjäljitystä ja verrataan niitä Rustin vastaaviin ominaisuuksiin.

C ja \Cpp ovat siirrettäviä (engl. portable) lähdekoodin näkökulmasta, eli sama lähdekoodi voidaan kääntää mille tahansa alustalle, mikäli koodissa ei käytetä jollekin tietylle alustalle tai järjestelmälle ominaisia kirjastoja. Lisäksi tarvitaan kääntäjä, joka tukee haluttua alustaa. Suosituimmat C- ja \Cpp-kääntäjät ovat GCC eli GNU Compiler Collectioniin sisältyvät gcc ja g++, jotka tukevat monia alustoja~\cite{gcctarget}.

Rust on C-kielien tavoin mahdollista kääntää eri alustoille ja sen rustc-kääntäjä tukeekin monia eri alustoja. Tuettujen alustojen määrä on kuitenkin C-kieliä pienempi ja tuki saattaa vaihdella. Rustissa tuki alustoille on jaettu kolmeen eri tasoon sen mukaan, kuinka paljon kyseisiä alustoja on testattu ja onko Rustin omassa kääntäjässä virallinen tuki niille~\cite{rustc}. Rustin kääntäjä on avointa lähdekoodia ja joillekin virallista tukea vailla oleville alustoille on kehitetty omia kääntäjiä esimerkiksi laitevalmistajan toimesta. Tällaisia ovat esimerkiksi Xtensa-arkkitehtuuriin perustuvat prosessorit, joille on saatavilla Espressif Systemsin kehittämä kääntäjä~\cite{esprs}.

Suoritusnopeudeltaan Rust on samalla tasolla C ja \Cpp-kielien kanssa~\cite{rustvsc} eli se soveltuu tältä osin käytettäväksi samoissa, nopeutta vaativissa sovelluksissa, esimerkiksi sulautetuissa järjestelmissä ja laiteajureissa. Rust sisältää C-kielien tapaan vain minimaalisen suoritusaikaisen ympäristön (engl. runtime) ja mahdollistaa näin korkean ja ennakoitavissa olevan suorituskyvyn sekä pienen muistijalanjäljen. Rustia voidaan käyttää myös esimerkiksi roskienkerääjän implementoinnissa, jossa sen nopeus vastaa C-kielellä tehtyä viiteimplementaatiota~\cite[p.~94]{rustgc}. Rustc tarjoaa gcc:n tavoin eri tasoisia käännösvaiheen optimointeja. Molemmista kääntäjistä löytyy optimointitasot 0-3, joilla saa otettua käyttöön ohjelman suoritusnopeutta lisääviä käännösasetuksia. Lisäksi kääntäjässä on s- ja z-tasot, joilla voi optimoida ohjelman kokoa pienemmäksi. Tällöin ohjelman suoritusnopeus saattaa kuitenkin jäädä huonommaksi.

Rustin kääntäjä käyttää C-kielien Clang kääntäjän tapaan LLVM-pohjaa, minkä johdosta Rust on tuettu monissa C- ja \Cpp-kielien kanssa käytetyissä virheenjäljitystyökaluissa. Esimerkiksi Visual Studio debugger, The GNU project debugger ja LLVM-projektiin kuuluva LLDB Debugger tukevat Rustia.

\section{Kirjastot} \label{kirjastot}
C- ja \Cpp-kielille on pitkän ikänsä aikana kehitetty monia kirjastoja ja moniin ongelmiin löytyykin usein valmis ratkaisu. Myös monet kolmannen osapuolen palveluita ylläpitävät tahot, kuten tietokantaohjelmistojen valmistajat, ovat tehneet kirjastoja C-kielille, joilla palvelujen käyttö onnistuu helposti. C-kielillä kirjaston asentaminen tapahtuu pääasiassa manuaalisesti lataamalla kirjasto ja lisäämällä se projektin tiedostoihin. C- ja \Cpp-kielille on myös kehitetty Conan-niminen avoimen lähdekoodin pakettihallintaohjelma, jonka avulla kirjastojen lataamista ja käyttöönottoa on pyritty helpottamaan. C-kielien kirjastoille ei kuitenkaan ole kehittynyt Rustin crates.io:n tai Javascriptin npm-rekisterin tapaista laajaa ekosysteemiä~\cite{WOS:000449166500015}.

Rustille on myös kehitetty jo monia kirjastoja, tosin määrä on C-kieliä pienempi. Rustilla on myös C-kieliä keskitetympi tapa kirjastojen hallintaan, minkä ansiosta kirjastojen löytäminen ja käyttäminen on helpompaa. Rustin asennuksen mukana tulee Cargo-pakettihallintaohjelma, jonka kautta kirjastoja voi asentaa riippuvuuksina ohjelmaan ja päivittää haluttaessa myöhemmin. Rustin käyttö tapahtuukin pääasiassa Cargo-ohjelman kautta alakomenoilla eikä suoraan kääntäjää kutsumalla. Esimerkiksi ohjelman käännetään yleensä \textit{cargo build} -komennolla \textit{rustc}-komennon sijaan, jolloin Cargo lataa ja kääntää kaikki ohjelman riippuvuudet sekä itse ohjelman. 

Rustin kirjastot löytyvät pääasiassa crates.io-palvelusta. Rustilla kirjaston käyttöönotto tapahtuu lisäämällä se Cargo.toml-nimiseen tiedostoon ja asettamalla Semver-syntaksia käyttämällä versionumero sekä mihin saakka riippuvuuden versiota voidaan päivittää~\cite{cargo}. Cargo-ohjelman kautta on myös mahdollista asentaa Rustilla kirjoitettuja ohjelmia \textit{cargo install} -komennolla. 